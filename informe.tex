%%%%%%%%%%%%%%%%%%%%%%%%%%%%%%%%%%%%%%%%%%%%%%%%%%%%%%%%%%%%%%%%%%%%%%%%%%%%%%%%
%
%	Copyright 2018 
%	Juli�n Nonino
%
%%%%%%%%%%%%%%%%%%%%%%%%%%%%%%%%%%%%%%%%%%%%%%%%%%%%%%%%%%%%%%%%%%%%%%%%%%%%%%%%
\documentclass[a4paper,12pt,openright,twoside]{book}

% PAQUETES	
	% Idioma y codificaci�n de caracteres
		\usepackage[spanish]{babel}
		\usepackage[latin1]{inputenc}
	% Figuras
		\usepackage{graphicx}
		\graphicspath{{img/}}
		\usepackage{subfigure}
		\usepackage{float} % Para posicionar im�genes donde uno quiera. Solo hay que poner la opci�n [H]
	% Ap�ndices
		\usepackage{appendix}
	%Tablas	
		%\usepackage{tabular}
	% Margenes
		\usepackage{anysize}
	% Tabla de contenido
		\usepackage[tight]{shorttoc}
	% Matem�tica
		\usepackage[cmex10]{amsmath}
		\usepackage{amssymb}
	% Referencias
		\usepackage{csquotes}
		\usepackage{hyperref}
		\usepackage[sorting=none,backend=bibtex]{biblatex}
		\addbibresource{referencias.bib}
	% Colores
		\usepackage{color}
		\definecolor{dkgreen}{rgb}{0,0.6,0}
		\definecolor{gray}{rgb}{0.5,0.5,0.5}
		\definecolor{mauve}{rgb}{0.58,0,0.82}
		\definecolor{violeta}{RGB}{127,0,85}
	% Insertar c�digo
		\usepackage{listings}
	% Margenes {izquierda}{derecha}{arriba}{abajo}
		\marginsize{3cm}{3cm}{2.5cm}{2.5cm}
	% Encabezados
		\pagestyle{headings}
		
% DOCUMENTO
\begin{document}
 
	% COMANDOS
		\renewcommand{\appendixname}{Ap�ndice}
		\renewcommand{\appendixtocname}{Ap�ndice}
		\renewcommand{\tablename}{\textbf{Tabla}}
		\renewcommand{\figurename}{\textbf{Figura}}
		\renewcommand{\contentsname}{�ndice}
		\renewcommand{\listtablename}{�ndice de tablas}
		\renewcommand{\listfigurename}{�ndice de Figuras}
		\setcounter{secnumdepth}{3} % Para numerar sub sub secciones
		\setcounter{tocdepth}{3} % Para incluir sub sub secciones en la ToC

	% PRIMERA PARTE
	% - Portada
	% - Tabla de Contenido Reducida
	\frontmatter
		%%%%%%%%%%%%%%%%%%%%%%%%%%%%%%%%%%%%%%%%%%%%%%%%%%%%%%%%%%%%%%%%%%%%%%%%%%%%%%%%
%	Copyright 2018 
%	Juli�n Nonino
%%%%%%%%%%%%%%%%%%%%%%%%%%%%%%%%%%%%%%%%%%%%%%%%%%%%%%%%%%%%%%%%%%%%%%%%%%%%%%%%
% PAGINA ANTERIOR
\begin{titlepage}
    \thispagestyle{empty}
    \center

    \rule{\linewidth}{0.5mm}
    \\[0.5cm]
    \LARGE \textbf{Teor�a Musical}
    \\
    \Large Apuntes y Ejercicios para Guitarra
    \\
    \rule{\linewidth}{0.5mm}
    \\[0.5cm]
    
    \vfill  \vfill

    \Large{
        \textit{Autor}
        \\
        Juli�n \textsc{Nonino}
    }

    \vfill  \vfill  \vfill  \vfill  \vfill  \vfill  \vfill
 
% PAGINA POSTERIOR
    \newpage
    \mbox{}
    \thispagestyle{empty}

\end{titlepage}
		\shorttableofcontents{Tabla de contenido}{0}

	% TEXTO PRINCIPAL
	\mainmatter
	\part{Introducci�n}
		   %%%%%%%%%%%%%%%%%%%%%%%%%%%%%%%%%%%%%%%%%%%%%%%%%%%%%%%%%%%%%%%%%%%%%%%%%%%%%%%%
%	Copyright 2018 
%	Juli�n Nonino
%%%%%%%%%%%%%%%%%%%%%%%%%%%%%%%%%%%%%%%%%%%%%%%%%%%%%%%%%%%%%%%%%%%%%%%%%%%%%%%%
\chapter{Introducci�n}
\label{chap:introduccion}

\section{Conceptos}
        
PARA COMPLETAR
    
\section{Notas musicales}

PARA COMPLETAR

    \subsection{Las notas en la guitarra}

    PARA COMPLETAR

\section{Introducci�n a los intervalos}

PARA COMPLETAR
		   %%%%%%%%%%%%%%%%%%%%%%%%%%%%%%%%%%%%%%%%%%%%%%%%%%%%%%%%%%%%%%%%%%%%%%%%%%%%%%%%
%	Copyright 2018 
%	Juli�n Nonino
%%%%%%%%%%%%%%%%%%%%%%%%%%%%%%%%%%%%%%%%%%%%%%%%%%%%%%%%%%%%%%%%%%%%%%%%%%%%%%%%
\chapter{Notaci�n musical}
\label{chap:chap_notacion_musical}
    
PARA COMPLETAR

\section{Introducci�n}
\label{sec:introduccion_notacion_musical}

PARA COMPLETAR

\section{Pentagrama}

PARA COMPLETAR

    \subsection{Claves y notas en el pentagrama}

    PARA COMPLETAR
        
    \subsection{Alteraciones y armaduras}

    PARA COMPLETAR
    
    \subsection{Sistemas}

    PARA COMPLETAR

\section{Tablaturas}

PARA COMPLETAR
		   %%%%%%%%%%%%%%%%%%%%%%%%%%%%%%%%%%%%%%%%%%%%%%%%%%%%%%%%%%%%%%%%%%%%%%%%%%%%%%%%
%	Copyright 2018 
%	Juli�n Nonino
%%%%%%%%%%%%%%%%%%%%%%%%%%%%%%%%%%%%%%%%%%%%%%%%%%%%%%%%%%%%%%%%%%%%%%%%%%%%%%%%
\chapter{Valores R�tmicos}
\label{chap:chap_valores_ritmicos}
    
PARA COMPLETAR

\section{Introducci�n}
\label{sec:introduccion_valores_ritmicos}

PARA COMPLETAR
	\part{Armon�a}
		%%%%%%%%%%%%%%%%%%%%%%%%%%%%%%%%%%%%%%%%%%%%%%%%%%%%%%%%%%%%%%%%%%%%%%%%%%%%%%%%
%	Copyright 2018 
%	Juli�n Nonino
%%%%%%%%%%%%%%%%%%%%%%%%%%%%%%%%%%%%%%%%%%%%%%%%%%%%%%%%%%%%%%%%%%%%%%%%%%%%%%%%
\chapter{Escalas}
\label{chap:escalas}

PARA COMPLETAR

\section{Introducci�n}

PARA COMPLETAR

\section{Modos}

PARA COMPLETAR

    \subsection{J�nico}
        
    PARA COMPLETAR

    \subsection{D�rico}

    PARA COMPLETAR

    \subsection{Frigio}

    PARA COMPLETAR

    \subsection{Lidio}

    PARA COMPLETAR

    \subsection{Mixolidio}

    \subsection{E�lico}

    \subsection{Locrio}

\section{Escala pentat�nica}

PARA COMPLETAR
    
\section{C�rculo de quintas}
    
PARA COMPLETAR
    
		%%%%%%%%%%%%%%%%%%%%%%%%%%%%%%%%%%%%%%%%%%%%%%%%%%%%%%%%%%%%%%%%%%%%%%%%%%%%%%%%
%	Copyright 2018 
%	Juli�n Nonino
%%%%%%%%%%%%%%%%%%%%%%%%%%%%%%%%%%%%%%%%%%%%%%%%%%%%%%%%%%%%%%%%%%%%%%%%%%%%%%%%
\chapter{Intervalos}
\label{chap:intervalos}
		%%%%%%%%%%%%%%%%%%%%%%%%%%%%%%%%%%%%%%%%%%%%%%%%%%%%%%%%%%%%%%%%%%%%%%%%%%%%%%%%
%	Copyright 2018 
%	Juli�n Nonino
%%%%%%%%%%%%%%%%%%%%%%%%%%%%%%%%%%%%%%%%%%%%%%%%%%%%%%%%%%%%%%%%%%%%%%%%%%%%%%%%
\chapter{Acordes}
\label{chap:acordes}

PARA COMPLETAR

\section{Tr�adas}

PARA COMPLETAR

\section{Inversiones}

PARA COMPLETAR
	%\part{Desarrollo}
		%\input{./desarrollo/diseno_implementacion/chap_diseno_implementacion}

	% PARTE FINAL. Bibliograf�a e indices
	% - Bibliograf�a
	% - �ndice completo de contenido
	% - �ndice de figuras
	\backmatter
		\printbibliography[heading=bibnumbered, title={Bibliograf�a}]
		\addcontentsline{toc}{chapter}{�ndice de contenido}
		\tableofcontents
		\cleardoublepage
		\addcontentsline{toc}{chapter}{�ndice de Figuras}
		\listoffigures

\end{document}