%%%%%%%%%%%%%%%%%%%%%%%%%%%%%%%%%%%%%%%%%%%%%%%%%%%%%%%%%%%%%%%%%%%%%%%%%%%%%%%%
%
%	Copyright 2018 
%	Juli�n Nonino
%
%%%%%%%%%%%%%%%%%%%%%%%%%%%%%%%%%%%%%%%%%%%%%%%%%%%%%%%%%%%%%%%%%%%%%%%%%%%%%%%%
\documentclass[a4paper,12pt,openright,twoside]{book}

% PAQUETES	
	% Idioma y codificaci�n de caracteres
		\usepackage[spanish]{babel}
		\usepackage[latin1]{inputenc}
	% Figuras
		\usepackage{graphicx}
		\graphicspath{{img/}}
		\usepackage{subfigure}
		\usepackage{float} % Para posicionar im�genes donde uno quiera. Solo hay que poner la opci�n [H]
	% Ap�ndices
		\usepackage{appendix}
	%Tablas	
		%\usepackage{tabular}
	% Margenes
		\usepackage{anysize}
	% Tabla de contenido
		\usepackage[tight]{shorttoc}
	% Matem�tica
		\usepackage[cmex10]{amsmath}
		\usepackage{amssymb}
	% Referencias
		\usepackage{csquotes}
		\usepackage{hyperref}
		\usepackage[sorting=none,backend=bibtex]{biblatex}
		\addbibresource{referencias.bib}
	% Colores
		\usepackage{color}
		\definecolor{dkgreen}{rgb}{0,0.6,0}
		\definecolor{gray}{rgb}{0.5,0.5,0.5}
		\definecolor{mauve}{rgb}{0.58,0,0.82}
		\definecolor{violeta}{RGB}{127,0,85}
	% Insertar c�digo
		\usepackage{listings}
	% Margenes {izquierda}{derecha}{arriba}{abajo}
		\marginsize{3cm}{3cm}{2.5cm}{2.5cm}
	% Encabezados
		\pagestyle{headings}
		
% DOCUMENTO
\begin{document}
 
	% COMANDOS
		\renewcommand{\appendixname}{Ap�ndice}
		\renewcommand{\appendixtocname}{Ap�ndice}
		\renewcommand{\tablename}{\textbf{Tabla}}
		\renewcommand{\figurename}{\textbf{Figura}}
		\renewcommand{\contentsname}{�ndice}
		\renewcommand{\listtablename}{�ndice de tablas}
		\renewcommand{\listfigurename}{�ndice de Figuras}
		\setcounter{secnumdepth}{3} % Para numerar sub sub secciones
		\setcounter{tocdepth}{3} % Para incluir sub sub secciones en la ToC

	% PRIMERA PARTE
	% - Portada
	% - Tabla de Contenido Reducida
	\frontmatter
		%%%%%%%%%%%%%%%%%%%%%%%%%%%%%%%%%%%%%%%%%%%%%%%%%%%%%%%%%%%%%%%%%%%%%%%%%%%%%%%%
%	Copyright 2018 
%	Juli�n Nonino
%%%%%%%%%%%%%%%%%%%%%%%%%%%%%%%%%%%%%%%%%%%%%%%%%%%%%%%%%%%%%%%%%%%%%%%%%%%%%%%%
% PAGINA ANTERIOR
\begin{titlepage}
    \thispagestyle{empty}
    \center

    \rule{\linewidth}{0.5mm}
    \\[0.5cm]
    \LARGE \textbf{Teor�a Musical}
    \\
    \Large Apuntes y Ejercicios para Guitarra
    \\
    \rule{\linewidth}{0.5mm}
    \\[0.5cm]
    
    \vfill  \vfill

    \Large{
        \textit{Autor}
        \\
        Juli�n \textsc{Nonino}
    }

    \vfill  \vfill  \vfill  \vfill  \vfill  \vfill  \vfill
 
% PAGINA POSTERIOR
    \newpage
    \mbox{}
    \thispagestyle{empty}

\end{titlepage}
		\shorttableofcontents{Tabla de contenido}{0}

	% TEXTO PRINCIPAL
	\mainmatter
	\part{Introducci�n}
		   %%%%%%%%%%%%%%%%%%%%%%%%%%%%%%%%%%%%%%%%%%%%%%%%%%%%%%%%%%%%%%%%%%%%%%%%%%%%%%%%
%	Copyright 2018 
%	Juli�n Nonino
%%%%%%%%%%%%%%%%%%%%%%%%%%%%%%%%%%%%%%%%%%%%%%%%%%%%%%%%%%%%%%%%%%%%%%%%%%%%%%%%
\chapter{Introducci�n}
\label{chap:introduccion}
        
\section{Notas musicales}

El sonido es una onda generada por un cuerpo sonoro al vibrar trasmitida por alg�n medio, generalmente el aire, que luego es captada por el o�do.


Los sonidos musicales, generalmente se asocian a ondas sinusoidales sin cambios bruscos. Se utilizan las notas musicales para hacer referencia a un sonido de una determinada frecuencia. 

Aunque el o�do humano es capaz de percibir sonidos desde los 20 Hz y los 20 KHz, el lenguaje establecido consta de 12 notas musicales asociadas cada una a una frecuencia de vibraci�n. Si bien existen 12 notas musicales, la nomenclatura definida o el abecedario musical se compone de 7 nombres diferenciados, conocidos como notas fundamentales y el resto se forma a partir de alteraciones en los nombres anteriores.

\begin{table}[H]
    \centering
    \begin{tabular}{l|l|l|l|l|l|l|l|}
    \cline{2-8}
    Sistema Latino    & Do & Re & Mi & Fa & Sol & La & Si \\ \cline{2-8} 
    Cifrado Americano & C  & D  & E  & F  & G   & A  & B  \\ \cline{2-8} 
    \end{tabular}
\end{table}

Las alteraciones, pueden ser \textbf{\emph{sostenidos (\#)}} o \textbf{\emph{bemoles (b)}}, se agregan a las notas fundamentales para conformar el abecedario de 12 notas cada una de ellas separada por un valor entero conocido como semitono. Los semitonos son medidos desde el \emph{La Central ($La_4$)} el cual se dice que corresponde a la frecuencia de 440 Hz. la f�rmula para determinar la frecuencia de cada una de las notas musicales es la siguiente. Siendo \emph{n} un n�mero entero de semitonos, positivo para notas m�s agudas que el La Central y negativo para notas m�s graves.

\begin{equation*}
    f = 2^{n/12} \times 440 Hz
\end{equation*}

\begin{table}[H]
    \centering
    \begin{tabular}{|l|l|l|l|l|}
    \cline{1-4}
    Do          & C        & $f = 2^{-9/12} \times 440 Hz$ & $261.63 Hz$ \\ \cline{1-4}
    Do\# / Reb  & C\# / Db & $f = 2^{-8/12} \times 440 Hz$ & $277.18 Hz$ \\ \cline{1-4}
    Re          & D        & $f = 2^{-7/12} \times 440 Hz$ & $293.66 Hz$ \\ \cline{1-4}
    Re\# / Mib  & D\# / Eb & $f = 2^{-6/12} \times 440 Hz$ & $311.13 Hz$ \\ \cline{1-4}
    Mi          & E        & $f = 2^{-5/12} \times 440 Hz$ & $329.63 Hz$ \\ \cline{1-4}
    Fa          & F        & $f = 2^{-4/12} \times 440 Hz$ & $349.23 Hz$ \\ \cline{1-4}
    Fa\# / Sol  & F\# / Gb & $f = 2^{-3/12} \times 440 Hz$ & $369.99 Hz$ \\ \cline{1-4}
    Sol         & G        & $f = 2^{-2/12} \times 440 Hz$ & $392.00 Hz$ \\ \cline{1-4}
    Sol\# / Lab & G\# / Ab & $f = 2^{-1/12} \times 440 Hz$ & $415.30 Hz$ \\ \cline{1-4}
    La          & A        & $f = 2^{0/12}  \times 440 Hz$ & $440.00 Hz$ \\ \cline{1-4}
    La\# / Sib  & A\# / Bb & $f = 2^{1/12}  \times 440 Hz$ & $466.16 Hz$ \\ \cline{1-4}
    Si          & B        & $f = 2^{2/12}  \times 440 Hz$ & $493.88 Hz$ \\ \cline{1-4}
    Do          & C        & $f = 2^{3/12}  \times 440 Hz$ & $523.25 Hz$ \\ \cline{1-4}
    \end{tabular}
\end{table}

Como se puede observar, existen notas que reciben dos nombres, es decir, hay dos maneras diferentes de nombrar sonidos iguales. Esto se conoce como \textbf{\emph{enarmon�a}}. De todas maneras, aunque al mismo sonido se le puedan atribuir dos o m�s nombres, musicalmente no es lo mismo utilizar uno u otro nombre, como se ver� m�s adelante.

Luego de agotar los 12 s�mbolos, 12 semitonos mas adelante o detr�s, la nomenclatura comienza a repetirse. Esto se conoce como \textbf{\emph{octavas}}. Por ejemplo:

\begin{table}[H]
    \centering
    \begin{tabular}{|l|l|l|l|}
    \cline{1-4}
    La & A & $f = 2^{-24/12} \times 440 Hz$ & $= 110.00  Hz$ \\ \cline{1-4}
    La & A & $f = 2^{-12/12} \times 440 Hz$ & $= 220.00  Hz$ \\ \cline{1-4}
    La & A & $f = 2^{0/12}   \times 440 Hz$ & $= 440.00  Hz$ \\ \cline{1-4}
    La & A & $f = 2^{12/12}  \times 440 Hz$ & $= 880.00  Hz$ \\ \cline{1-4}
    La & A & $f = 2^{24/12}  \times 440 Hz$ & $= 1760.00 Hz$ \\ \cline{1-4}
    \end{tabular}
\end{table}

Lo mismo puede ser aplicado para cualquier nota.

    \subsection{Las notas en la guitarra}

    Las notas presentes en una guitarra, pueden variar seg�n la afinaci�n elegida pero existe una afinaci�n est�ndar para guitarras de seis cuerdas. Dicha afinaci�n es, comenzando desde la cuerda m�s grave a la m�s aguda, Mi/E, La/A, D/Re, G/Sol, B/Si y E/Mi.
    
    Cada cuerda recibe un numero a modo de nombre de esta. De esta manera, la cuerda m�s grave, la superior es la sexta cuerda y la cuerda inferior, la m�s aguda, es la primera cuerda. En la siguiente tabla se muestra que nota se corresponde cada traste de la guitarra.

    \begin{table}[H]
        \centering
        \begin{tabular}{cccccccccccccc}
        \textit{\textbf{}}                & \textit{\textbf{Aire(0)}} & \textit{\textbf{1}}      & \textit{\textbf{2}}      & \textit{\textbf{3}}      & \textit{\textbf{4}}      & \textit{\textbf{5}}    & \textit{\textbf{6}}      & \textit{\textbf{7}}      & \textit{\textbf{8}}      & \textit{\textbf{9}}      & \textit{\textbf{10}}   & \textit{\textbf{11}}     & \textit{\textbf{12}}   \\ \cline{2-14} 
        \multicolumn{1}{c|}{\textit{1ra}} & \multicolumn{1}{c|}{E}    & \multicolumn{1}{c|}{F}   & \multicolumn{1}{c|}{F\#} & \multicolumn{1}{c|}{G}   & \multicolumn{1}{c|}{G\#} & \multicolumn{1}{c|}{A} & \multicolumn{1}{c|}{A\#} & \multicolumn{1}{c|}{B}   & \multicolumn{1}{c|}{C}   & \multicolumn{1}{c|}{C\#} & \multicolumn{1}{c|}{D} & \multicolumn{1}{c|}{D\#} & \multicolumn{1}{c|}{E} \\ \cline{2-14} 
        \multicolumn{1}{c|}{\textit{2da}} & \multicolumn{1}{c|}{B}    & \multicolumn{1}{c|}{C}   & \multicolumn{1}{c|}{C\#} & \multicolumn{1}{c|}{D}   & \multicolumn{1}{c|}{D\#} & \multicolumn{1}{c|}{E} & \multicolumn{1}{c|}{F}   & \multicolumn{1}{c|}{F\#} & \multicolumn{1}{c|}{G}   & \multicolumn{1}{c|}{G\#} & \multicolumn{1}{c|}{A} & \multicolumn{1}{c|}{A\#} & \multicolumn{1}{c|}{B} \\ \cline{2-14} 
        \multicolumn{1}{c|}{\textit{3ra}} & \multicolumn{1}{c|}{G}    & \multicolumn{1}{c|}{G\#} & \multicolumn{1}{c|}{A}   & \multicolumn{1}{c|}{A\#} & \multicolumn{1}{c|}{B}   & \multicolumn{1}{c|}{C} & \multicolumn{1}{c|}{C\#} & \multicolumn{1}{c|}{D}   & \multicolumn{1}{c|}{D\#} & \multicolumn{1}{c|}{E}   & \multicolumn{1}{c|}{F} & \multicolumn{1}{c|}{F\#} & \multicolumn{1}{c|}{G} \\ \cline{2-14} 
        \multicolumn{1}{c|}{\textit{4ta}} & \multicolumn{1}{c|}{D}    & \multicolumn{1}{c|}{D\#} & \multicolumn{1}{c|}{E}   & \multicolumn{1}{c|}{F}   & \multicolumn{1}{c|}{F\#} & \multicolumn{1}{c|}{G} & \multicolumn{1}{c|}{G\#} & \multicolumn{1}{c|}{A}   & \multicolumn{1}{c|}{A\#} & \multicolumn{1}{c|}{B}   & \multicolumn{1}{c|}{C} & \multicolumn{1}{c|}{C\#} & \multicolumn{1}{c|}{D} \\ \cline{2-14} 
        \multicolumn{1}{c|}{\textit{5ta}} & \multicolumn{1}{c|}{A}    & \multicolumn{1}{c|}{A\#} & \multicolumn{1}{c|}{B}   & \multicolumn{1}{c|}{C}   & \multicolumn{1}{c|}{C\#} & \multicolumn{1}{c|}{D} & \multicolumn{1}{c|}{D\#} & \multicolumn{1}{c|}{E}   & \multicolumn{1}{c|}{F}   & \multicolumn{1}{c|}{F\#} & \multicolumn{1}{c|}{G} & \multicolumn{1}{c|}{G\#} & \multicolumn{1}{c|}{A} \\ \cline{2-14} 
        \multicolumn{1}{c|}{\textit{6ta}} & \multicolumn{1}{c|}{E}    & \multicolumn{1}{c|}{F}   & \multicolumn{1}{c|}{F\#} & \multicolumn{1}{c|}{G}   & \multicolumn{1}{c|}{G\#} & \multicolumn{1}{c|}{A} & \multicolumn{1}{c|}{A\#} & \multicolumn{1}{c|}{B}   & \multicolumn{1}{c|}{C}   & \multicolumn{1}{c|}{C\#} & \multicolumn{1}{c|}{D} & \multicolumn{1}{c|}{D\#} & \multicolumn{1}{c|}{E} \\ \cline{2-14} 
        \end{tabular}
    \end{table}

    Con esta afinaci�n se puede conseguir abarcar tres octavas, como se observa en la siguiente figura.

    \begin{figure}[H]
		\centering
		\includegraphics[width=1\linewidth,keepaspectratio]{NotasGuitarra}
		\caption{Octavas en la guitarra}
		\label{fig:cronograma}
    \end{figure}
    
    Existen algunas reglas �tiles para ubicar y memorizar las notas en el m�stil de la guitarra.
    \begin{enumerate}
        \item La primera cuerda y la sexta tienen las mismas notas.
        \item Desde cualquier nota en la sexta cuerda, yendo hasta la cuarta cuerda (cuerda de por medio) y a un traste de por medio hacia adelante se encuentra la octava.
        \item Desde cualquier nota en la cuarta cuerda, yendo hasta la segunda cuerda (cuerda de por medio) y a dos trastes de por medio hacia adelante se encuentra la octava.
        \item Desde cualquier nota en la primera cuerda, yendo hasta la tercera cuerda, dos trastes de por medio atr�s esta la octava.
        \item Desde cualquier nota en la tercera cuerda, yendo hasta la quinta cuerda, un traste de por medio atr�s esta la octava.
    \end{enumerate}

\section{Introducci�n a los intervalos}

Un intervalo musical es la diferencia de \textbf{altura (frecuencia)} entre dos notas musicales, medida \textbf{cuantitativamente (n�mero)} en grados o notas naturales y \textbf{cualitativamente (especie)} en tonos y semitonos. Los intervalos musicales tienen dos palabras que conforman su nombre completo, por ejemplo, un intervalo es: Segunda Mayor. La palabra 'segunda' es la primera parte que indica la clasificaci�n del intervalo (cantidad) y la palabra 'mayor' es la calificaci�n del intervalo (cualidad).

Existen 8 clasificaciones (cantidades) b�sicas para describir un intervalo: \textbf{Un�sono}, \textbf{Segunda}, \textbf{Tercera}, \textbf{Cuarta}, \textbf{Quinta}, \textbf{Sexta}, \textbf{S�ptima} y \textbf{Octava}. Para entender el concepto, es posible comenzar asumiendo que no existen las alteraciones y que s�lo existen las notas C, D, E, F, G, A y B. Para saber la clasificaci�n del intervalo entre dos notas, por ejemplo, entre E y A, se debe contar la cantidad de notas entre ellas: con E va una, con F van dos, con G van tres y con A son cuatro, por lo tanto, es una cuarta. Si el resultado fuese 5 entonces ser�a una quinta y as� sucesivamente. 

Cuando se introducen las alteraciones, aparecen dos nombres posibles para la misma nota se necesita una mayor precisi�n para determinar cual de los nombres corresponde utilizar. Por ejemplo, si se parte de la nota \emph{C}, y analizamos la nota \emph{F\#/Gb}, sabemos que la distancia entre ambas notas es de 6 semitonos pero surge la pregunta de si es una \textbf{\emph{cuarta (F\#)}} o una \textbf{\emph{quinta (Gb)}}.  Aqu� es donde son necesarias las cualidades de los intervalos. Para nombrar las cualidades se utilizan 5 palabras b�sicas: \textbf{Mayor}, \textbf{Menor}, \textbf{Aumentada}, \textbf{Disminuida} y \textbf{Justa}. Estas cualidades indican la distancia en tonos exacta entre dos notas.

\begin{table}[H]
    \centering
    \begin{tabular}{ccccc}
    \textbf{Clasificaci�n} & \multicolumn{4}{c}{\textbf{Calificaci�n}}                          \\
    Segunda                & Disminuida (0st)  & Menor (1st)  & Mayor (2st)  & Aumentada (3st)  \\
    Tercera                & Disminuida (2st)  & Menor (3st)  & Mayor (4st)  & Aumentada (5st)  \\
    Cuarta                 & Disminuida (4st)  &              & Justa (5st)  & Aumentada (6st)  \\
    Quinta                 & Disminuida (6st)  &              & Justa (7st)  & Aumentada (8st)  \\
    Sexta                  & Disminuida (7st)  & Menor (8st)  & Mayor (9st)  & Aumentada (10st) \\
    S�ptima                & Disminuida (9st)  & Menor (10st) & Mayor (11st) & Aumentada (12st) \\
    Octava                 & Disminuida (11sp) &              & Justa (12st) & Aumentada (13st)
    \end{tabular}
\end{table}

De esta manera, si se habla de una cuarta aumentada por encima de \emph{C}, se habla exactamente de un \emph{F\#} y no de un \emph{Gb}. Del mismo modo, indicar una quinta disminuida por encima de \emph{C} es nombrar la nota \emph{Gb}, no \emph{F\#}.

Los \emph{aumentados} son un semitono m�s alto que un intervalo \emph{mayor}, y son un semitono m�s alto que un intervalo \emph{justo}. Los intervalos \emph{disminuidos} son un semitono m�s peque�os que un intervalo \emph{justo}, y un semitono m�s peque�os que un intervalo \emph{menor}. Por ejemplo:

\begin{table}[H]
    \centering
    \begin{tabular}{lllllll}
    \hline
    \multicolumn{2}{l}{Nota Base}           & \multicolumn{5}{l}{C}                                                                                   \\ \hline
    \multicolumn{7}{l}{}                                                                                                                              \\ \hline
    Sexta & \multicolumn{1}{l|}{Disminuida} & \multicolumn{1}{l|}{Abb} & \multicolumn{1}{l|}{7 st}  & \multicolumn{1}{l|}{G}   & Justa      & Quinta  \\ \hline
    Sexta & \multicolumn{1}{l|}{Menor}      & \multicolumn{1}{l|}{Ab}  & \multicolumn{1}{l|}{8 st}  & \multicolumn{1}{l|}{G\#} & Aumentada  & Quinta  \\ \hline
    Sexta & \multicolumn{1}{l|}{Mayor}      & \multicolumn{1}{l|}{A}   & \multicolumn{1}{l|}{9 st}  & \multicolumn{1}{l|}{Bbb} & Disminuida & S�ptima \\ \hline
    Sexta & \multicolumn{1}{l|}{Aumentado}  & \multicolumn{1}{l|}{A\#} & \multicolumn{1}{l|}{10 st} & \multicolumn{1}{l|}{Bb}  & Menor      & S�ptima \\ \hline
    \end{tabular}
\end{table}
		   %%%%%%%%%%%%%%%%%%%%%%%%%%%%%%%%%%%%%%%%%%%%%%%%%%%%%%%%%%%%%%%%%%%%%%%%%%%%%%%%
%	Copyright 2018 
%	Juli�n Nonino
%%%%%%%%%%%%%%%%%%%%%%%%%%%%%%%%%%%%%%%%%%%%%%%%%%%%%%%%%%%%%%%%%%%%%%%%%%%%%%%%
\chapter{Notaci�n musical}
\label{chap:chap_notacion_musical}
    
PARA COMPLETAR

\section{Introducci�n}
\label{sec:introduccion_notacion_musical}

PARA COMPLETAR

\section{Pentagrama}

PARA COMPLETAR

    \subsection{Claves y notas en el pentagrama}

    PARA COMPLETAR
        
    \subsection{Alteraciones y armaduras}

    PARA COMPLETAR
    
    \subsection{Sistemas}

    PARA COMPLETAR

\section{Tablaturas}

PARA COMPLETAR
		   %%%%%%%%%%%%%%%%%%%%%%%%%%%%%%%%%%%%%%%%%%%%%%%%%%%%%%%%%%%%%%%%%%%%%%%%%%%%%%%%
%	Copyright 2018 
%	Juli�n Nonino
%%%%%%%%%%%%%%%%%%%%%%%%%%%%%%%%%%%%%%%%%%%%%%%%%%%%%%%%%%%%%%%%%%%%%%%%%%%%%%%%
\chapter{Valores R�tmicos}
\label{chap:chap_valores_ritmicos}
    
PARA COMPLETAR

\section{Introducci�n}
\label{sec:introduccion_valores_ritmicos}

PARA COMPLETAR
	\part{Armon�a}
		%%%%%%%%%%%%%%%%%%%%%%%%%%%%%%%%%%%%%%%%%%%%%%%%%%%%%%%%%%%%%%%%%%%%%%%%%%%%%%%%
%	Copyright 2018 
%	Juli�n Nonino
%%%%%%%%%%%%%%%%%%%%%%%%%%%%%%%%%%%%%%%%%%%%%%%%%%%%%%%%%%%%%%%%%%%%%%%%%%%%%%%%
\chapter{Escalas}
\label{chap:escalas}

PARA COMPLETAR

\section{Introducci�n}

PARA COMPLETAR

\section{Modos}

PARA COMPLETAR

    \subsection{J�nico}
        
    PARA COMPLETAR

    \subsection{D�rico}

    PARA COMPLETAR

    \subsection{Frigio}

    PARA COMPLETAR

    \subsection{Lidio}

    PARA COMPLETAR

    \subsection{Mixolidio}

    \subsection{E�lico}

    \subsection{Locrio}

\section{Escala pentat�nica}

PARA COMPLETAR
    
\section{C�rculo de quintas}
    
PARA COMPLETAR
    
		%%%%%%%%%%%%%%%%%%%%%%%%%%%%%%%%%%%%%%%%%%%%%%%%%%%%%%%%%%%%%%%%%%%%%%%%%%%%%%%%
%	Copyright 2018 
%	Juli�n Nonino
%%%%%%%%%%%%%%%%%%%%%%%%%%%%%%%%%%%%%%%%%%%%%%%%%%%%%%%%%%%%%%%%%%%%%%%%%%%%%%%%
\chapter{Intervalos}
\label{chap:intervalos}
		%%%%%%%%%%%%%%%%%%%%%%%%%%%%%%%%%%%%%%%%%%%%%%%%%%%%%%%%%%%%%%%%%%%%%%%%%%%%%%%%
%	Copyright 2018 
%	Juli�n Nonino
%%%%%%%%%%%%%%%%%%%%%%%%%%%%%%%%%%%%%%%%%%%%%%%%%%%%%%%%%%%%%%%%%%%%%%%%%%%%%%%%
\chapter{Acordes}
\label{chap:acordes}

PARA COMPLETAR

\section{Tr�adas}

PARA COMPLETAR

\section{Inversiones}

PARA COMPLETAR
	%\part{Desarrollo}
		%\input{./desarrollo/diseno_implementacion/chap_diseno_implementacion}

	% PARTE FINAL. Bibliograf�a e indices
	% - Bibliograf�a
	% - �ndice completo de contenido
	% - �ndice de figuras
	\backmatter
		\printbibliography[heading=bibnumbered, title={Bibliograf�a}]
		\addcontentsline{toc}{chapter}{�ndice de contenido}
		\tableofcontents
		\cleardoublepage
		\addcontentsline{toc}{chapter}{�ndice de Figuras}
		\listoffigures

\end{document}